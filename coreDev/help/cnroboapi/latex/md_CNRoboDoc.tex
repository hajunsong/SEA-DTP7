\subsection*{\hyperlink{classCNRobo}{C\-N\-Robo} Header 및 Library}

\section*{사용 방법}

\section*{1) C\-N\-Robo의 주요 header file \-:}

\begin{DoxyVerb}    cnrobo.h : robot api를 제공한다.
    cntype.h : robot api에서 사용하는 type 정의 
    cnerror.h : error code 정의
    cnhelper.h : 각종 함수 제공.
\end{DoxyVerb}


\section*{2) C\-N\-Robo의 library file \-: libcnrobo – Release파일, libcnrobo – Debug파일}

\begin{DoxyVerb}    linux86 : libcnrobo.a, libcnrobo_d.a
    linuxarm : libcnrobo.a, libcnrobo_d.a
    mingw : libcnrobo.a, libcnrobo_d.a
    msvc : cnrobo.lib, cnrobo_d.lib
\end{DoxyVerb}


\section*{3) Cui\-Api 구조}

\begin{DoxyVerb}    로봇의 제어할 수 있는 Main Program인 corecon이 있고 corecon에 내장되어있는
    Robot Engine을 제어할 수 있는 CUIAPI가 있다. CUIAPI를 통해 Custom UI에서는 
    Robot의 각종 기능들을 corecon ui와는 별도로 제어할 수 있다.
    Corecon과 Custom UI를 Plugin 하는 과정이 필요한데, Custom UI를 
    So library 형태로 제작하여 corecon에 Custom UI에 대한 Widget Pointer를 Return하여 
    corecon에서 Custom UI의 Widget를 Show 하여 사용한다.\end{DoxyVerb}
 